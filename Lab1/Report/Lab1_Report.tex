\documentclass[12pt,a4paper,table]{article}
\usepackage[a4paper,
            tmargin=2cm,
            bmargin=2cm,
            lmargin=2cm,
            rmargin=2cm,
            bindingoffset=0cm]{geometry}


\usepackage{lmodern}
\usepackage[T1]{polski}
\usepackage[utf8]{inputenc}
\usepackage{tocloft}
\usepackage{hyperref}
\usepackage{amsmath}
\usepackage{listings}
\usepackage{graphicx}
\usepackage{subfig}
\usepackage{float}
\usepackage{booktabs}

\hypersetup{
    colorlinks,
    citecolor=black,
    filecolor=black,
    linkcolor=black,
    urlcolor=black
}

\newtheorem{definition}{Def}

\begin{document}
    \title{
        Informatyka Medyczna \\
        Laboratorium 1
    }
    \author{
        Adam Staniszewski \\
        Ewa Żukowska
    }

    \date{\today}

    \maketitle

    \tableofcontents
    \newpage
    Celem laboratorium było przygotowanie narzędzia, które umożliwiałoby śledzenie położenia wkłuwanej igły. Dzięki wizualizacji odszumionego sygnału możemy zaobserwować moment wkłucia.
    \section{Zadanie 1}
    W zadaniu pierwszym należało dodać możliwość zastosowania filtra Gaussaw celu odszumienia sygnału.
    \begin{lstlisting}[language=Python]
    from scipy.ndimage import gaussian_filter
    
    frames_filtered = gaussian_filter(frames, sigma=5)
    \end{lstlisting}

    \vspace{1em}
    \raggedright
    Następnie trzeba było dodać filtr Butterwortha, który posłuży do dalszego odszumiania. 

    \begin{lstlisting}[language=Python]
    @staticmethod
    def butter_bandpass(lowcut, highcut, fs, order=5):
        nyquist = 0.5 * fs
        low = lowcut / nyquist
        high = highcut / nyquist
        b, a = butter(order, [low, high], btype='band')
        return b, a

    @staticmethod
    def butter_bandpass_filter(data, lowcut, highcut, fs, order=5):
        b, a = Inka.butter_bandpass(lowcut, highcut, fs, order=order)
        y = lfilter(b, a, data)
        return y
    \end{lstlisting}

    \vspace{1em}
    \raggedright
    Ostatnim krokiem w tym zadaniu było prygotowanie wizualizacji dotyczących poprzednich punktów.

    \begin{lstlisting}[language=Python]
        TUTAJ KOD
    \end{lstlisting}
    
    \section{Zadanie 2}
    W zadaniu drugim należało 
    \section{Zadanie 3}
    W zadaniu trzecim

\end{document}
